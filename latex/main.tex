% DOCUMENT SETUP %

\documentclass[twocolumn]{article}
\usepackage[italian]{babel}

\usepackage{tgschola}
\linespread{1.25}
\usepackage[fontsize=12pt]{scrextend}
\usepackage[a4paper,top=2.5cm,bottom=3cm,left=3cm,right=3cm,marginparwidth=1.75cm,footskip=1.75cm,heightrounded]{geometry}

\usepackage{xcolor}
\definecolor{linkColor}{RGB}{2,11,120}
\usepackage[colorlinks=true, allcolors=linkColor]{hyperref}

\usepackage{graphicx}
\usepackage{amsmath}
\usepackage{float}

\usepackage{subfig}
\graphicspath{ {./images/} }

% HEADERS %

\title{Relazione Laboratorio Algoritmi}
\author{Alberto Del Buono Paolini}
\date{Ottobre 2023}

\begin{document}

\begin{onecolumn}
\maketitle
\tableofcontents
\vspace{3cm}
\end{onecolumn}
\pagebreak

% CONTENT %

\section{Introduzione}

Questo studio mira a confrontare diverse implementazioni di statistiche d'ordine dinamiche, mettendo a confronto tre approcci principali: utilizzo di una lista ordinata, di un Albero Binario di Ricerca (ABR) senza attributo \textit{size}, e di un ABR come discusso nelle lezioni. L'obiettivo è valutare le prestazioni e le caratteristiche di ciascuna implementazione attraverso test ed esperimenti pratici.

\section{Caratteristiche teoriche}

La soluzione più performante tra le tre opzioni dipenderà da vari fattori, tra cui il tipo di operazioni che intendi eseguire con maggiore frequenza, la dimensione dei dati con cui stai lavorando e le specifiche esigenze del tuo problema. Di seguito fornisco alcune considerazioni generali:

\subsection{Lista Ordinata}

L'implementazione con lista ordinata utilizza una struttura dati basata su una lista concatenata con puntatori. In questa struttura, gli elementi vengono inseriti in modo che siano mantenuti in ordine crescente o decrescente, a seconda della specifica implementazione. Questo comporta che l'inserimento di un nuovo elemento richieda l'individuazione della posizione corretta in cui inserirlo in modo che l'ordine sia conservato. Le operazioni di ricerca in una lista ordinata sono relativamente efficienti poiché è possibile sfruttare l'ordine degli elementi per effettuare ricerche binarie, riducendo il tempo di ricerca rispetto a una lista non ordinata.

Tuttavia, l'inserimento e la rimozione possono essere più costosi in termini di tempo rispetto a una struttura dati come un ABR, in quanto richiedono spostamenti e aggiornamenti dei puntatori per mantenere l'ordine. La complessità di queste operazioni può dipendere dalla dimensione della lista e dalla posizione dell'elemento da inserire o rimuovere.

\subsection{Albero Binario di Ricerca senza attributo \textit{size}}

L'Albero Binario di Ricerca (ABR) è una struttura dati in cui ogni nodo ha al massimo due figli: uno a sinistra contenente valori inferiori e uno a destra contenente valori superiori rispetto al nodo padre. L'ABR senza l'attributo \textit{size} utilizza questa struttura per organizzare gli elementi in modo che gli elementi più piccoli siano situati nei sottoalberi sinistri e quelli più grandi nei sottoalberi destri.

Le operazioni di inserimento, ricerca e rimozione in un ABR senza attributo \textit{size} dipendono dalla struttura dell'albero. In particolare, la loro complessità può variare in base all'equilibrio dell'albero. Se l'albero è bilanciato, le operazioni possono essere efficienti, ma se l'albero è sbilanciato, la complessità può aumentare.

\subsection{Albero Rosso-Nero con attributo \textit{size}}

Un Albero Rosso-Nero con attributo \textit{size} introduce un miglioramento significativo nelle operazioni di ricerca basate su statistiche d'ordine. L'inserimento in un Albero Rosso-Nero con attributo \textit{size} è simile a quello di un ABR senza attributo \textit{size}, ma con un piccolo overhead aggiuntivo per l'aggiornamento dell'attributo \textit{size} dei nodi durante l'inserimento. La complessità rimane $O(\log n)$ nel caso migliore (con albero bilanciato).

La ricerca di statistiche d'ordine come il k-esimo elemento più piccolo o il rango di un elemento diventa molto efficiente in un Albero Rosso-Nero con attributo \textit{size}. Queste operazioni possono essere eseguite in $O(\log n)$, il che è notevolmente più veloce rispetto a una lista ordinata o a un ABR senza attributo \textit{size}.

In generale, ci si aspetta che un Albero Rosso-Nero con attributo \textit{size} eccella nelle operazioni di ricerca basate su statistiche d'ordine, mentre potrebbe essere leggermente meno efficiente nelle operazioni di inserimento e rimozione rispetto a un ABR senza attributo \textit{size}. Tuttavia, l'efficienza dipende anche dalla struttura dell'albero e dalla sua capacità di mantenere un buon bilanciamento. Gli esperimenti permetteranno di confermare queste previsioni e fornire dati empirici sulle prestazioni.

\section{Descrizione degli esperimenti}

In questa sezione, delineeremo in dettaglio gli esperimenti che verranno condotti. Non ci limiteremo a un semplice elenco, ma spiegheremo il motivo dietro ciascun esperimento e come verrà misurato il suo successo.

\subsection{Strategia di iterazione}

Puoi eseguire iterazioni su un insieme di dati in diversi modi, ad esempio:

\begin{itemize}
  \item \textbf{Iterazione casuale}: Esegui le operazioni in modo casuale sugli elementi dell'insieme di dati. Questo è utile per valutare le prestazioni in scenari realistici in cui le operazioni potrebbero essere distribuite in modo casuale nel tempo.
  \item \textbf{Iterazione crescente o decrescente}: Esegui le operazioni in ordine crescente o decrescente rispetto ai valori degli elementi. Questo può aiutarti a valutare le prestazioni quando le operazioni seguono un ordine specifico.
  \item \textbf{Iterazione mista}: Combina l'iterazione casuale con l'iterazione crescente o decrescente per coprire una varietà di scenari.
\end{itemize}

\subsection{Dimensioni delle strutture dati}

Esegui i tuoi test su diverse dimensioni delle strutture dati per valutare come le prestazioni variano con l'aumento del numero di elementi nell'insieme di dati. Ad esempio, puoi testare le prestazioni su insiemi di dati di dimensioni 10, 100, 1000, ecc.

Assicurati di considerare sia casi di piccole dimensioni che casi di grandi dimensioni, poiché le prestazioni possono differire notevolmente in base alle dimensioni dell'insieme di dati.

\subsection{Ripetizioni dei test}

Per ottenere risultati affidabili, esegui ciascun test più volte e calcola la media dei tempi di esecuzione. Questo ridurrà l'impatto di variazioni casuali e fornirà risultati più stabili.

Puoi anche calcolare la deviazione standard o l'intervallo di confidenza per misurare la variabilità dei risultati.

\subsection{Randomizzazione}

L'uso di dati casuali durante i test può essere utile per valutare le prestazioni in scenari realistici, ma assicurati di registrare i semi (seed) dei generatori casuali utilizzati nei test in modo da poter riprodurre i risultati in futuro, se necessario.

\section{Documentazione del codice}

Forniremo documentazione soprattutto delle implementazioni delle statistiche d'ordine dinamiche per tutte e tre le strutture dati.

\section{Analisi dei risultati sperimentali}

In questa sezione, condurremo un'analisi completa dei risultati sperimentali. Discuteremo le discrepanze tra le prestazioni attese e quelle effettivamente osservate, se presenti, e offriremo interpretazioni basate sui dati raccolti.

Riferimenti alle tabelle: \ref{Inserimento in una lista ordinata}, \ref{Rimozione in una lista ordinata}, \ref{Ricerca in una lista ordinata}, \ref{OS-Select in una lista ordinata}, \ref{OS-Rank in una lista ordinata}, \ref{Inserimento in un ABR}, \ref{Rimozione in un ABR}, \ref{Ricerca in un ABR}, \ref{OS-Select in un ABR}, \ref{OS-Rank in un ABR}, \ref{Inserimento in un albero RN aumentato}, \ref{Rimozione in un albero RN aumentato}, \ref{Ricerca in un albero RN aumentato}, \ref{OS-Select in un albero RN aumentato}, \ref{OS-Rank in un albero RN aumentato}.

\section{Conclusioni}

Concluderemo la relazione con un riepilogo delle scoperte chiave e delle implicazioni dei risultati sperimentali. Discuteremo anche eventuali sviluppi futuri e possibili miglioramenti.

\newpage
\begin{table}
\centering
\caption{OS-Select in una lista ordinata}
\label{OS-Select in una lista ordinata}
\begin{adjustbox}{width=1\textwidth/2}
\begin{tabular}{|c|c|c|c|}
\hline
 & Dimensione (n) & Mediana (s) & Tempi -/+ (s) \\
0 & 10000 & 4.1e-04 & -4.1e-04, +9.7e-04 \\
\cline{1-4}
1 & 10500 & 4.6e-04 & -4.6e-04, +8.2e-04 \\
\cline{1-4}
2 & 11000 & 4.8e-04 & -4.8e-04, +2.2e-03 \\
\cline{1-4}
3 & 11500 & 5.0e-04 & -5.0e-04, +1.8e-03 \\
\cline{1-4}
4 & 12000 & 4.9e-04 & -4.9e-04, +9.2e-04 \\
\cline{1-4}
5 & 12500 & 5.9e-04 & -5.8e-04, +4.3e-03 \\
\cline{1-4}
6 & 13000 & 5.6e-04 & -5.6e-04, +3.7e-03 \\
\cline{1-4}
7 & 13500 & 5.8e-04 & -5.8e-04, +1.0e-03 \\
\cline{1-4}
8 & 14000 & 6.1e-04 & -6.1e-04, +1.1e-03 \\
\cline{1-4}
9 & 14500 & 6.3e-04 & -6.2e-04, +1.1e-03 \\
\cline{1-4}
10 & 15000 & 6.5e-04 & -6.5e-04, +3.7e-03 \\
\cline{1-4}
11 & 15500 & 6.7e-04 & -6.6e-04, +2.7e-03 \\
\cline{1-4}
12 & 16000 & 7.1e-04 & -7.1e-04, +2.7e-03 \\
\cline{1-4}
13 & 16500 & 7.1e-04 & -7.1e-04, +3.0e-03 \\
\cline{1-4}
14 & 17000 & 7.4e-04 & -7.4e-04, +3.6e-03 \\
\cline{1-4}
15 & 17500 & 7.8e-04 & -7.8e-04, +2.3e-03 \\
\cline{1-4}
16 & 18000 & 8.6e-04 & -8.6e-04, +3.3e-03 \\
\cline{1-4}
17 & 18500 & 8.6e-04 & -8.6e-04, +5.4e-03 \\
\cline{1-4}
18 & 19000 & 8.2e-04 & -8.2e-04, +1.9e-03 \\
\cline{1-4}
19 & 19500 & 9.3e-04 & -9.3e-04, +8.2e-03 \\
\cline{1-4}
20 & 20000 & 9.3e-04 & -9.3e-04, +5.5e-03 \\
\cline{1-4}
\end{tabular}
\end{adjustbox}
\end{table}

\begin{table}
\centering
\caption{OS-Rank in una lista ordinata}
\label{OS-Rank in una lista ordinata}
\begin{adjustbox}{width=1\textwidth/2}
\begin{tabular}{|c|c|c|c|}
\hline
 & Dimensione (n) & Mediana (s) & Tempi -/+ (s) \\
0 & 10000 & 4.7e-04 & -4.7e-04, +1.4e-03 \\
\cline{1-4}
1 & 10500 & 5.1e-04 & -5.1e-04, +9.3e-04 \\
\cline{1-4}
2 & 11000 & 5.7e-04 & -5.7e-04, +3.2e-03 \\
\cline{1-4}
3 & 11500 & 5.8e-04 & -5.8e-04, +4.6e-03 \\
\cline{1-4}
4 & 12000 & 5.5e-04 & -5.4e-04, +3.2e-03 \\
\cline{1-4}
5 & 12500 & 6.4e-04 & -6.3e-04, +5.5e-03 \\
\cline{1-4}
6 & 13000 & 6.3e-04 & -6.3e-04, +2.1e-03 \\
\cline{1-4}
7 & 13500 & 6.6e-04 & -6.6e-04, +4.2e-03 \\
\cline{1-4}
8 & 14000 & 7.3e-04 & -7.3e-04, +1.3e-03 \\
\cline{1-4}
9 & 14500 & 7.1e-04 & -7.1e-04, +1.4e-03 \\
\cline{1-4}
10 & 15000 & 8.0e-04 & -8.0e-04, +6.2e-03 \\
\cline{1-4}
11 & 15500 & 8.0e-04 & -8.0e-04, +1.8e-03 \\
\cline{1-4}
12 & 16000 & 8.3e-04 & -8.3e-04, +2.7e-03 \\
\cline{1-4}
13 & 16500 & 8.1e-04 & -8.1e-04, +3.0e-03 \\
\cline{1-4}
14 & 17000 & 8.9e-04 & -8.9e-04, +8.4e-03 \\
\cline{1-4}
15 & 17500 & 8.2e-04 & -8.1e-04, +4.2e-03 \\
\cline{1-4}
16 & 18000 & 1.0e-03 & -1.0e-03, +7.1e-03 \\
\cline{1-4}
17 & 18500 & 9.1e-04 & -9.1e-04, +3.7e-03 \\
\cline{1-4}
18 & 19000 & 9.8e-04 & -9.7e-04, +4.6e-03 \\
\cline{1-4}
19 & 19500 & 1.2e-03 & -1.2e-03, +1.3e-02 \\
\cline{1-4}
20 & 20000 & 1.1e-03 & -1.1e-03, +4.3e-03 \\
\cline{1-4}
\end{tabular}
\end{adjustbox}
\end{table}

\begin{table}
\centering
\caption{OS-Select in un ABR}
\label{OS-Select in un ABR}
\begin{adjustbox}{width=1\textwidth/2}
\begin{tabular}{|c|c|c|c|}
\hline
 & Dimensione (n) & Mediana (s) & Tempi -/+ (s) \\
0 & 10 & 1.4e-06 & -5.3e-07, +6.1e-06 \\
\cline{1-4}
1 & 20 & 2.6e-06 & -1.5e-06, +5.3e-06 \\
\cline{1-4}
2 & 30 & 4.0e-06 & -3.0e-06, +2.4e-05 \\
\cline{1-4}
3 & 40 & 4.1e-06 & -2.9e-06, +3.5e-05 \\
\cline{1-4}
4 & 50 & 8.9e-06 & -4.8e-06, +2.4e-05 \\
\cline{1-4}
5 & 60 & 1.0e-05 & -5.2e-06, +3.6e-05 \\
\cline{1-4}
6 & 70 & 8.3e-06 & -5.1e-06, +3.1e-05 \\
\cline{1-4}
7 & 80 & 1.1e-05 & -9.5e-06, +3.6e-05 \\
\cline{1-4}
8 & 90 & 1.4e-05 & -8.9e-06, +4.0e-02 \\
\cline{1-4}
9 & 100 & 1.9e-05 & -1.0e-05, +4.4e-05 \\
\cline{1-4}
10 & 110 & 1.2e-05 & -7.6e-06, +3.1e-05 \\
\cline{1-4}
11 & 120 & 2.1e-05 & -1.8e-05, +3.4e-05 \\
\cline{1-4}
12 & 130 & 3.1e-05 & -2.7e-05, +3.6e-05 \\
\cline{1-4}
13 & 140 & 2.3e-05 & -1.3e-05, +4.0e-05 \\
\cline{1-4}
14 & 150 & 2.8e-05 & -1.7e-05, +6.6e-05 \\
\cline{1-4}
15 & 160 & 2.4e-05 & -1.4e-05, +4.3e-05 \\
\cline{1-4}
16 & 170 & 3.2e-05 & -1.8e-05, +5.1e-05 \\
\cline{1-4}
17 & 180 & 3.1e-05 & -2.5e-05, +5.7e-05 \\
\cline{1-4}
18 & 190 & 4.3e-05 & -2.6e-05, +5.0e-05 \\
\cline{1-4}
19 & 200 & 2.0e-05 & -1.4e-05, +6.1e-05 \\
\cline{1-4}
20 & 210 & 4.1e-05 & -4.0e-05, +1.4e-04 \\
\cline{1-4}
21 & 220 & 2.8e-05 & -2.7e-05, +5.5e-05 \\
\cline{1-4}
22 & 230 & 4.5e-05 & -2.5e-05, +4.7e-05 \\
\cline{1-4}
23 & 240 & 2.5e-05 & -1.4e-05, +5.6e-05 \\
\cline{1-4}
24 & 250 & 3.8e-05 & -3.8e-05, +1.2e-02 \\
\cline{1-4}
25 & 260 & 2.0e-05 & -1.7e-05, +3.7e-05 \\
\cline{1-4}
26 & 270 & 4.3e-05 & -2.3e-05, +4.6e-05 \\
\cline{1-4}
27 & 280 & 4.8e-05 & -2.6e-05, +4.3e-05 \\
\cline{1-4}
28 & 290 & 6.3e-05 & -4.2e-05, +2.4e-04 \\
\cline{1-4}
29 & 300 & 7.8e-05 & -6.8e-05, +1.0e-04 \\
\cline{1-4}
30 & 310 & 2.9e-05 & -2.0e-05, +7.6e-05 \\
\cline{1-4}
31 & 320 & 3.2e-05 & -2.2e-05, +4.5e-05 \\
\cline{1-4}
32 & 330 & 5.4e-05 & -3.0e-05, +8.4e-05 \\
\cline{1-4}
33 & 340 & 5.0e-05 & -2.5e-05, +1.0e-04 \\
\cline{1-4}
34 & 350 & 3.3e-05 & -2.3e-05, +6.6e-05 \\
\cline{1-4}
35 & 360 & 3.4e-05 & -2.7e-05, +5.9e-05 \\
\cline{1-4}
36 & 370 & 5.3e-05 & -2.7e-05, +6.5e-05 \\
\cline{1-4}
37 & 380 & 5.6e-05 & -4.9e-05, +5.9e-05 \\
\cline{1-4}
38 & 390 & 1.1e-04 & -7.4e-05, +1.5e-04 \\
\cline{1-4}
39 & 400 & 7.8e-05 & -4.8e-05, +8.6e-05 \\
\cline{1-4}
\end{tabular}
\end{adjustbox}
\end{table}

\begin{table}
\centering
\begin{adjustbox}{width=1\textwidth/2}
\begin{tabular}{|c|c|c|c|}
\hline
 & Dimensione (n) & Mediana (s) & Tempi -/+ (s) \\
40 & 410 & 3.8e-05 & -2.1e-05, +6.3e-05 \\
\cline{1-4}
41 & 420 & 5.8e-05 & -3.7e-05, +6.8e-05 \\
\cline{1-4}
42 & 430 & 4.0e-05 & -3.2e-05, +9.7e-05 \\
\cline{1-4}
43 & 440 & 7.5e-05 & -6.3e-05, +1.1e-04 \\
\cline{1-4}
44 & 450 & 8.1e-05 & -5.1e-05, +9.1e-05 \\
\cline{1-4}
45 & 460 & 5.9e-05 & -4.8e-05, +1.1e-04 \\
\cline{1-4}
46 & 470 & 1.1e-04 & -6.7e-05, +9.8e-05 \\
\cline{1-4}
47 & 480 & 5.6e-05 & -4.7e-05, +8.2e-05 \\
\cline{1-4}
48 & 490 & 6.9e-05 & -4.0e-05, +8.8e-05 \\
\cline{1-4}
49 & 500 & 6.2e-05 & -3.2e-05, +7.2e-05 \\
\cline{1-4}
50 & 510 & 6.0e-05 & -4.0e-05, +2.1e-04 \\
\cline{1-4}
51 & 520 & 5.8e-05 & -3.9e-05, +6.1e-05 \\
\cline{1-4}
52 & 530 & 1.1e-04 & -7.0e-05, +1.5e-04 \\
\cline{1-4}
53 & 540 & 5.0e-05 & -2.9e-05, +1.3e-04 \\
\cline{1-4}
54 & 550 & 1.4e-04 & -8.9e-05, +1.4e-04 \\
\cline{1-4}
55 & 560 & 1.5e-04 & -9.8e-05, +3.5e-04 \\
\cline{1-4}
56 & 570 & 8.2e-05 & -3.9e-05, +8.2e-05 \\
\cline{1-4}
57 & 580 & 8.2e-05 & -4.1e-05, +1.3e-04 \\
\cline{1-4}
58 & 590 & 6.5e-05 & -4.9e-05, +2.6e-04 \\
\cline{1-4}
59 & 600 & 7.5e-05 & -4.9e-05, +8.9e-05 \\
\cline{1-4}
60 & 610 & 9.8e-05 & -9.3e-05, +9.9e-05 \\
\cline{1-4}
61 & 620 & 1.0e-04 & -6.3e-05, +1.0e-04 \\
\cline{1-4}
62 & 630 & 9.3e-05 & -5.7e-05, +1.3e-04 \\
\cline{1-4}
63 & 640 & 6.1e-05 & -2.8e-05, +7.5e-05 \\
\cline{1-4}
64 & 650 & 6.3e-05 & -5.1e-05, +8.9e-05 \\
\cline{1-4}
65 & 660 & 1.2e-04 & -7.6e-05, +3.1e-04 \\
\cline{1-4}
66 & 670 & 1.0e-04 & -6.0e-05, +1.5e-04 \\
\cline{1-4}
67 & 680 & 1.4e-04 & -7.8e-05, +2.3e-04 \\
\cline{1-4}
68 & 690 & 1.0e-04 & -6.1e-05, +8.3e-05 \\
\cline{1-4}
69 & 700 & 1.2e-04 & -6.6e-05, +1.4e-04 \\
\cline{1-4}
70 & 710 & 1.1e-04 & -5.3e-05, +1.9e-04 \\
\cline{1-4}
71 & 720 & 9.8e-05 & -9.5e-05, +1.2e-04 \\
\cline{1-4}
72 & 730 & 1.1e-04 & -6.2e-05, +1.9e-04 \\
\cline{1-4}
73 & 740 & 1.2e-04 & -6.9e-05, +5.6e-02 \\
\cline{1-4}
74 & 750 & 8.8e-05 & -8.0e-05, +1.6e-04 \\
\cline{1-4}
75 & 760 & 9.5e-05 & -5.1e-05, +1.1e-04 \\
\cline{1-4}
76 & 770 & 1.4e-04 & -7.3e-05, +1.8e-04 \\
\cline{1-4}
77 & 780 & 1.1e-04 & -9.0e-05, +2.0e-02 \\
\cline{1-4}
78 & 790 & 1.1e-04 & -9.0e-05, +2.1e-04 \\
\cline{1-4}
79 & 800 & 1.1e-04 & -7.4e-05, +2.6e-04 \\
\cline{1-4}
\end{tabular}
\end{adjustbox}
\end{table}

\begin{table}
\centering
\begin{adjustbox}{width=1\textwidth/2}
\begin{tabular}{|c|c|c|c|}
\hline
 & Dimensione (n) & Mediana (s) & Tempi -/+ (s) \\
80 & 810 & 7.8e-05 & -4.3e-05, +6.8e-05 \\
\cline{1-4}
81 & 820 & 7.8e-05 & -7.4e-05, +2.0e-04 \\
\cline{1-4}
82 & 830 & 1.4e-04 & -8.9e-05, +1.3e-04 \\
\cline{1-4}
83 & 840 & 7.5e-05 & -4.7e-05, +5.4e-05 \\
\cline{1-4}
84 & 850 & 1.7e-04 & -9.7e-05, +1.9e-04 \\
\cline{1-4}
85 & 860 & 9.7e-05 & -7.3e-05, +1.5e-04 \\
\cline{1-4}
86 & 870 & 1.2e-04 & -6.7e-05, +1.1e-04 \\
\cline{1-4}
87 & 880 & 1.2e-04 & -9.1e-05, +1.1e-04 \\
\cline{1-4}
88 & 890 & 9.1e-05 & -3.6e-05, +2.0e-04 \\
\cline{1-4}
89 & 900 & 1.6e-04 & -8.7e-05, +1.4e-04 \\
\cline{1-4}
90 & 910 & 1.0e-04 & -7.5e-05, +2.1e-04 \\
\cline{1-4}
91 & 920 & 2.9e-04 & -2.1e-04, +4.3e-04 \\
\cline{1-4}
92 & 930 & 1.0e-04 & -7.4e-05, +2.5e-04 \\
\cline{1-4}
93 & 940 & 1.9e-04 & -1.1e-04, +5.5e-04 \\
\cline{1-4}
94 & 950 & 1.3e-04 & -8.3e-05, +3.1e-04 \\
\cline{1-4}
95 & 960 & 1.4e-04 & -8.8e-05, +2.8e-04 \\
\cline{1-4}
96 & 970 & 9.0e-05 & -6.9e-05, +2.8e-04 \\
\cline{1-4}
97 & 980 & 1.8e-04 & -1.0e-04, +3.0e-04 \\
\cline{1-4}
98 & 990 & 1.4e-04 & -8.3e-05, +1.6e-04 \\
\cline{1-4}
99 & 1000 & 1.3e-04 & -8.7e-05, +4.3e-04 \\
\cline{1-4}
\end{tabular}
\end{adjustbox}
\end{table}

\begin{table}
\centering
\caption{OS-Rank in un ABR}
\label{OS-Rank in un ABR}
\begin{adjustbox}{width=1\textwidth/2}
\begin{tabular}{|c|c|c|c|}
\hline
 & Dimensione (n) & Mediana (s) & Tempi -/+ (s) \\
0 & 10 & 9.2e-07 & -5.3e-07, +4.1e-06 \\
\cline{1-4}
1 & 20 & 1.5e-06 & -1.1e-06, +1.8e-05 \\
\cline{1-4}
2 & 30 & 2.0e-06 & -1.5e-06, +5.8e-06 \\
\cline{1-4}
3 & 40 & 2.5e-06 & -2.0e-06, +2.2e-05 \\
\cline{1-4}
4 & 50 & 2.8e-06 & -2.4e-06, +3.0e-05 \\
\cline{1-4}
5 & 60 & 3.6e-06 & -2.9e-06, +2.5e-05 \\
\cline{1-4}
6 & 70 & 3.9e-06 & -3.2e-06, +2.1e-05 \\
\cline{1-4}
7 & 80 & 5.7e-06 & -5.1e-06, +2.4e-05 \\
\cline{1-4}
8 & 90 & 7.1e-06 & -6.4e-06, +3.6e-05 \\
\cline{1-4}
9 & 100 & 5.4e-06 & -4.6e-06, +2.8e-05 \\
\cline{1-4}
10 & 110 & 6.0e-06 & -5.4e-06, +2.0e-05 \\
\cline{1-4}
11 & 120 & 6.7e-06 & -6.2e-06, +3.5e-05 \\
\cline{1-4}
12 & 130 & 7.1e-06 & -6.3e-06, +1.7e-05 \\
\cline{1-4}
13 & 140 & 7.4e-06 & -6.7e-06, +2.4e-05 \\
\cline{1-4}
14 & 150 & 8.1e-06 & -7.4e-06, +2.6e-05 \\
\cline{1-4}
15 & 160 & 9.6e-06 & -8.8e-06, +4.6e-05 \\
\cline{1-4}
16 & 170 & 9.7e-06 & -8.8e-06, +2.0e-05 \\
\cline{1-4}
17 & 180 & 1.1e-05 & -1.0e-05, +3.5e-05 \\
\cline{1-4}
18 & 190 & 1.2e-05 & -1.1e-05, +2.6e-05 \\
\cline{1-4}
19 & 200 & 1.1e-05 & -9.9e-06, +3.0e-05 \\
\cline{1-4}
20 & 210 & 1.3e-05 & -1.3e-05, +5.1e-05 \\
\cline{1-4}
21 & 220 & 1.1e-05 & -1.1e-05, +3.8e-05 \\
\cline{1-4}
22 & 230 & 1.2e-05 & -1.1e-05, +3.5e-05 \\
\cline{1-4}
23 & 240 & 1.2e-05 & -1.1e-05, +3.6e-05 \\
\cline{1-4}
24 & 250 & 1.3e-05 & -1.3e-05, +3.6e-05 \\
\cline{1-4}
25 & 260 & 1.3e-05 & -1.2e-05, +4.0e-05 \\
\cline{1-4}
26 & 270 & 1.3e-05 & -1.3e-05, +4.3e-05 \\
\cline{1-4}
27 & 280 & 1.4e-05 & -1.3e-05, +3.8e-05 \\
\cline{1-4}
28 & 290 & 1.5e-05 & -1.4e-05, +5.1e-05 \\
\cline{1-4}
29 & 300 & 2.1e-05 & -2.0e-05, +6.9e-05 \\
\cline{1-4}
30 & 310 & 1.5e-05 & -1.4e-05, +4.7e-05 \\
\cline{1-4}
31 & 320 & 1.9e-05 & -1.8e-05, +5.2e-05 \\
\cline{1-4}
32 & 330 & 1.8e-05 & -1.7e-05, +4.9e-05 \\
\cline{1-4}
33 & 340 & 1.8e-05 & -1.7e-05, +4.8e-05 \\
\cline{1-4}
34 & 350 & 1.9e-05 & -1.8e-05, +5.8e-05 \\
\cline{1-4}
35 & 360 & 1.8e-05 & -1.8e-05, +7.6e-05 \\
\cline{1-4}
36 & 370 & 1.9e-05 & -1.8e-05, +4.7e-05 \\
\cline{1-4}
37 & 380 & 2.0e-05 & -1.9e-05, +6.4e-05 \\
\cline{1-4}
38 & 390 & 1.9e-05 & -1.8e-05, +4.8e-05 \\
\cline{1-4}
39 & 400 & 1.9e-05 & -1.8e-05, +5.0e-05 \\
\cline{1-4}
\end{tabular}
\end{adjustbox}
\end{table}

\begin{table}
\centering
\begin{adjustbox}{width=1\textwidth/2}
\begin{tabular}{|c|c|c|c|}
\hline
 & Dimensione (n) & Mediana (s) & Tempi -/+ (s) \\
40 & 410 & 1.8e-05 & -1.7e-05, +4.4e-05 \\
\cline{1-4}
41 & 420 & 2.0e-05 & -1.9e-05, +7.4e-05 \\
\cline{1-4}
42 & 430 & 2.0e-05 & -1.9e-05, +5.5e-05 \\
\cline{1-4}
43 & 440 & 2.1e-05 & -2.1e-05, +5.1e-05 \\
\cline{1-4}
44 & 450 & 2.1e-05 & -2.0e-05, +6.6e-05 \\
\cline{1-4}
45 & 460 & 2.5e-05 & -2.4e-05, +7.0e-05 \\
\cline{1-4}
46 & 470 & 2.2e-05 & -2.1e-05, +6.3e-05 \\
\cline{1-4}
47 & 480 & 2.7e-05 & -2.6e-05, +7.7e-05 \\
\cline{1-4}
48 & 490 & 2.6e-05 & -2.5e-05, +5.1e-05 \\
\cline{1-4}
49 & 500 & 2.6e-05 & -2.5e-05, +6.9e-05 \\
\cline{1-4}
50 & 510 & 2.5e-05 & -2.4e-05, +1.9e-04 \\
\cline{1-4}
51 & 520 & 2.5e-05 & -2.4e-05, +6.3e-05 \\
\cline{1-4}
52 & 530 & 2.6e-05 & -2.5e-05, +7.8e-05 \\
\cline{1-4}
53 & 540 & 2.6e-05 & -2.5e-05, +5.6e-05 \\
\cline{1-4}
54 & 550 & 2.6e-05 & -2.5e-05, +8.5e-05 \\
\cline{1-4}
55 & 560 & 3.1e-05 & -2.9e-05, +1.6e-04 \\
\cline{1-4}
56 & 570 & 2.7e-05 & -2.6e-05, +6.4e-05 \\
\cline{1-4}
57 & 580 & 2.9e-05 & -2.8e-05, +2.4e-02 \\
\cline{1-4}
58 & 590 & 3.2e-05 & -3.1e-05, +1.1e-04 \\
\cline{1-4}
59 & 600 & 3.3e-05 & -3.2e-05, +9.3e-05 \\
\cline{1-4}
60 & 610 & 3.1e-05 & -3.1e-05, +5.4e-05 \\
\cline{1-4}
61 & 620 & 2.8e-05 & -2.7e-05, +6.6e-05 \\
\cline{1-4}
62 & 630 & 2.9e-05 & -2.8e-05, +7.7e-05 \\
\cline{1-4}
63 & 640 & 3.2e-05 & -3.1e-05, +7.7e-05 \\
\cline{1-4}
64 & 650 & 3.5e-05 & -3.4e-05, +6.9e-05 \\
\cline{1-4}
65 & 660 & 3.2e-05 & -3.1e-05, +1.1e-04 \\
\cline{1-4}
66 & 670 & 3.4e-05 & -3.2e-05, +9.8e-05 \\
\cline{1-4}
67 & 680 & 3.3e-05 & -3.2e-05, +5.5e-05 \\
\cline{1-4}
68 & 690 & 3.3e-05 & -3.3e-05, +7.0e-05 \\
\cline{1-4}
69 & 700 & 3.4e-05 & -3.4e-05, +7.4e-05 \\
\cline{1-4}
70 & 710 & 3.3e-05 & -3.2e-05, +5.4e-05 \\
\cline{1-4}
71 & 720 & 3.5e-05 & -3.4e-05, +9.5e-05 \\
\cline{1-4}
72 & 730 & 3.5e-05 & -3.4e-05, +1.4e-04 \\
\cline{1-4}
73 & 740 & 3.8e-05 & -3.6e-05, +8.7e-05 \\
\cline{1-4}
74 & 750 & 3.9e-05 & -3.8e-05, +1.2e-04 \\
\cline{1-4}
75 & 760 & 3.7e-05 & -3.6e-05, +8.1e-05 \\
\cline{1-4}
76 & 770 & 3.8e-05 & -3.7e-05, +8.2e-05 \\
\cline{1-4}
77 & 780 & 5.2e-05 & -5.1e-05, +1.1e-04 \\
\cline{1-4}
78 & 790 & 4.5e-05 & -4.4e-05, +1.3e-04 \\
\cline{1-4}
79 & 800 & 4.3e-05 & -4.2e-05, +1.9e-04 \\
\cline{1-4}
\end{tabular}
\end{adjustbox}
\end{table}

\begin{table}
\centering
\begin{adjustbox}{width=1\textwidth/2}
\begin{tabular}{|c|c|c|c|}
\hline
 & Dimensione (n) & Mediana (s) & Tempi -/+ (s) \\
80 & 810 & 3.9e-05 & -3.8e-05, +1.0e-04 \\
\cline{1-4}
81 & 820 & 4.3e-05 & -4.2e-05, +1.8e-04 \\
\cline{1-4}
82 & 830 & 3.7e-05 & -3.6e-05, +1.1e-04 \\
\cline{1-4}
83 & 840 & 3.7e-05 & -3.7e-05, +7.4e-05 \\
\cline{1-4}
84 & 850 & 4.1e-05 & -4.0e-05, +1.4e-04 \\
\cline{1-4}
85 & 860 & 4.1e-05 & -4.0e-05, +8.6e-05 \\
\cline{1-4}
86 & 870 & 4.4e-05 & -4.3e-05, +8.7e-05 \\
\cline{1-4}
87 & 880 & 4.2e-05 & -4.0e-05, +1.0e-04 \\
\cline{1-4}
88 & 890 & 4.3e-05 & -4.2e-05, +1.3e-04 \\
\cline{1-4}
89 & 900 & 4.2e-05 & -4.1e-05, +9.1e-05 \\
\cline{1-4}
90 & 910 & 4.8e-05 & -4.7e-05, +1.1e-04 \\
\cline{1-4}
91 & 920 & 4.8e-05 & -4.7e-05, +8.1e-03 \\
\cline{1-4}
92 & 930 & 4.7e-05 & -4.6e-05, +1.9e-04 \\
\cline{1-4}
93 & 940 & 4.6e-05 & -4.5e-05, +9.9e-05 \\
\cline{1-4}
94 & 950 & 4.7e-05 & -4.6e-05, +1.7e-04 \\
\cline{1-4}
95 & 960 & 4.9e-05 & -4.8e-05, +9.8e-05 \\
\cline{1-4}
96 & 970 & 5.0e-05 & -4.9e-05, +1.2e-04 \\
\cline{1-4}
97 & 980 & 4.8e-05 & -4.7e-05, +1.1e-04 \\
\cline{1-4}
98 & 990 & 4.9e-05 & -4.8e-05, +9.9e-05 \\
\cline{1-4}
99 & 1000 & 5.5e-05 & -5.3e-05, +1.7e-04 \\
\cline{1-4}
\end{tabular}
\end{adjustbox}
\end{table}

\begin{table}
\centering
\caption{OS-Select in un albero RN aumentato}
\label{OS-Select in un albero RN aumentato}
\begin{adjustbox}{width=1\textwidth/2}
\begin{tabular}{|c|c|c|c|}
\hline
 & Dimensione (n) & Tempo (s) & Deviazione standard \\
0 & 10 & 0.0000 & 0.0000 \\
\cline{1-4}
1 & 20 & 0.0000 & 0.0000 \\
\cline{1-4}
2 & 30 & 0.0000 & 0.0000 \\
\cline{1-4}
3 & 40 & 0.0000 & 0.0000 \\
\cline{1-4}
4 & 50 & 0.0000 & 0.0000 \\
\cline{1-4}
5 & 60 & 0.0000 & 0.0000 \\
\cline{1-4}
6 & 70 & 0.0001 & 0.0001 \\
\cline{1-4}
7 & 80 & 0.0001 & 0.0001 \\
\cline{1-4}
8 & 90 & 0.0001 & 0.0000 \\
\cline{1-4}
9 & 100 & 0.0001 & 0.0000 \\
\cline{1-4}
10 & 110 & 0.0001 & 0.0001 \\
\cline{1-4}
11 & 120 & 0.0001 & 0.0001 \\
\cline{1-4}
12 & 130 & 0.0001 & 0.0001 \\
\cline{1-4}
13 & 140 & 0.0001 & 0.0000 \\
\cline{1-4}
14 & 150 & 0.0001 & 0.0001 \\
\cline{1-4}
15 & 160 & 0.0001 & 0.0000 \\
\cline{1-4}
16 & 170 & 0.0002 & 0.0001 \\
\cline{1-4}
17 & 180 & 0.0002 & 0.0001 \\
\cline{1-4}
18 & 190 & 0.0002 & 0.0001 \\
\cline{1-4}
19 & 200 & 0.0002 & 0.0002 \\
\cline{1-4}
20 & 210 & 0.0002 & 0.0000 \\
\cline{1-4}
21 & 220 & 0.0002 & 0.0001 \\
\cline{1-4}
22 & 230 & 0.0002 & 0.0001 \\
\cline{1-4}
23 & 240 & 0.0002 & 0.0001 \\
\cline{1-4}
24 & 250 & 0.0003 & 0.0003 \\
\cline{1-4}
25 & 260 & 0.0003 & 0.0003 \\
\cline{1-4}
26 & 270 & 0.0003 & 0.0003 \\
\cline{1-4}
27 & 280 & 0.0003 & 0.0003 \\
\cline{1-4}
28 & 290 & 0.0004 & 0.0004 \\
\cline{1-4}
29 & 300 & 0.0003 & 0.0003 \\
\cline{1-4}
30 & 310 & 0.0003 & 0.0003 \\
\cline{1-4}
31 & 320 & 0.0004 & 0.0004 \\
\cline{1-4}
32 & 330 & 0.0004 & 0.0004 \\
\cline{1-4}
33 & 340 & 0.0004 & 0.0004 \\
\cline{1-4}
34 & 350 & 0.0004 & 0.0004 \\
\cline{1-4}
35 & 360 & 0.0004 & 0.0004 \\
\cline{1-4}
36 & 370 & 0.0004 & 0.0004 \\
\cline{1-4}
37 & 380 & 0.0005 & 0.0001 \\
\cline{1-4}
38 & 390 & 0.0005 & 0.0001 \\
\cline{1-4}
39 & 400 & 0.0005 & 0.0005 \\
\cline{1-4}
\end{tabular}
\end{adjustbox}
\end{table}

\begin{table}
\centering
\begin{adjustbox}{width=1\textwidth/2}
\begin{tabular}{|c|c|c|c|}
\hline
 & Dimensione (n) & Tempo (s) & Deviazione standard \\
40 & 410 & 0.0005 & 0.0002 \\
\cline{1-4}
41 & 420 & 0.0005 & 0.0005 \\
\cline{1-4}
42 & 430 & 0.0005 & 0.0002 \\
\cline{1-4}
43 & 440 & 0.0005 & 0.0005 \\
\cline{1-4}
44 & 450 & 0.0005 & 0.0002 \\
\cline{1-4}
45 & 460 & 0.0005 & 0.0005 \\
\cline{1-4}
46 & 470 & 0.0006 & 0.0006 \\
\cline{1-4}
47 & 480 & 0.0006 & 0.0001 \\
\cline{1-4}
48 & 490 & 0.0006 & 0.0006 \\
\cline{1-4}
49 & 500 & 0.0006 & 0.0006 \\
\cline{1-4}
50 & 510 & 0.0006 & 0.0006 \\
\cline{1-4}
51 & 520 & 0.0006 & 0.0006 \\
\cline{1-4}
52 & 530 & 0.0007 & 0.0007 \\
\cline{1-4}
53 & 540 & 0.0007 & 0.0007 \\
\cline{1-4}
54 & 550 & 0.0007 & 0.0007 \\
\cline{1-4}
55 & 560 & 0.0007 & 0.0007 \\
\cline{1-4}
56 & 570 & 0.0007 & 0.0007 \\
\cline{1-4}
57 & 580 & 0.0007 & 0.0003 \\
\cline{1-4}
58 & 590 & 0.0007 & 0.0003 \\
\cline{1-4}
59 & 600 & 0.0008 & 0.0004 \\
\cline{1-4}
60 & 610 & 0.0008 & 0.0008 \\
\cline{1-4}
61 & 620 & 0.0008 & 0.0006 \\
\cline{1-4}
62 & 630 & 0.0008 & 0.0007 \\
\cline{1-4}
63 & 640 & 0.0008 & 0.0008 \\
\cline{1-4}
64 & 650 & 0.0009 & 0.0009 \\
\cline{1-4}
65 & 660 & 0.0009 & 0.0009 \\
\cline{1-4}
66 & 670 & 0.0009 & 0.0009 \\
\cline{1-4}
67 & 680 & 0.0009 & 0.0009 \\
\cline{1-4}
68 & 690 & 0.0009 & 0.0009 \\
\cline{1-4}
69 & 700 & 0.0009 & 0.0009 \\
\cline{1-4}
70 & 710 & 0.0010 & 0.0010 \\
\cline{1-4}
71 & 720 & 0.0009 & 0.0009 \\
\cline{1-4}
72 & 730 & 0.0010 & 0.0010 \\
\cline{1-4}
73 & 740 & 0.0010 & 0.0010 \\
\cline{1-4}
74 & 750 & 0.0010 & 0.0010 \\
\cline{1-4}
75 & 760 & 0.0010 & 0.0010 \\
\cline{1-4}
76 & 770 & 0.0010 & 0.0010 \\
\cline{1-4}
77 & 780 & 0.0011 & 0.0011 \\
\cline{1-4}
78 & 790 & 0.0015 & 0.0015 \\
\cline{1-4}
79 & 800 & 0.0011 & 0.0009 \\
\cline{1-4}
\end{tabular}
\end{adjustbox}
\end{table}

\begin{table}
\centering
\begin{adjustbox}{width=1\textwidth/2}
\begin{tabular}{|c|c|c|c|}
\hline
 & Dimensione (n) & Tempo (s) & Deviazione standard \\
80 & 810 & 0.0013 & 0.0010 \\
\cline{1-4}
81 & 820 & 0.0013 & 0.0009 \\
\cline{1-4}
82 & 830 & 0.0013 & 0.0011 \\
\cline{1-4}
83 & 840 & 0.0014 & 0.0013 \\
\cline{1-4}
84 & 850 & 0.0013 & 0.0012 \\
\cline{1-4}
85 & 860 & 0.0013 & 0.0008 \\
\cline{1-4}
86 & 870 & 0.0014 & 0.0013 \\
\cline{1-4}
87 & 880 & 0.0013 & 0.0013 \\
\cline{1-4}
88 & 890 & 0.0016 & 0.0016 \\
\cline{1-4}
89 & 900 & 0.0018 & 0.0018 \\
\cline{1-4}
90 & 910 & 0.0013 & 0.0013 \\
\cline{1-4}
91 & 920 & 0.0013 & 0.0013 \\
\cline{1-4}
92 & 930 & 0.0013 & 0.0013 \\
\cline{1-4}
93 & 940 & 0.0015 & 0.0015 \\
\cline{1-4}
94 & 950 & 0.0016 & 0.0012 \\
\cline{1-4}
95 & 960 & 0.0014 & 0.0013 \\
\cline{1-4}
96 & 970 & 0.0016 & 0.0014 \\
\cline{1-4}
97 & 980 & 0.0014 & 0.0006 \\
\cline{1-4}
98 & 990 & 0.0015 & 0.0015 \\
\cline{1-4}
\end{tabular}
\end{adjustbox}
\end{table}

\begin{table}
\centering
\caption{OS-Rank in un albero RN aumentato}
\label{OS-Rank in un albero RN aumentato}
\begin{adjustbox}{width=1\textwidth/2}
\begin{tabular}{|c|c|c|c|}
\hline
 & Dimensione (n) & Tempo (s) & Deviazione standard \\
0 & 10 & 0.0000 & 0.0000 \\
\cline{1-4}
1 & 20 & 0.0000 & 0.0000 \\
\cline{1-4}
2 & 30 & 0.0000 & 0.0000 \\
\cline{1-4}
3 & 40 & 0.0000 & 0.0000 \\
\cline{1-4}
4 & 50 & 0.0000 & 0.0000 \\
\cline{1-4}
5 & 60 & 0.0000 & 0.0000 \\
\cline{1-4}
6 & 70 & 0.0000 & 0.0000 \\
\cline{1-4}
7 & 80 & 0.0001 & 0.0000 \\
\cline{1-4}
8 & 90 & 0.0001 & 0.0000 \\
\cline{1-4}
9 & 100 & 0.0001 & 0.0000 \\
\cline{1-4}
10 & 110 & 0.0001 & 0.0001 \\
\cline{1-4}
11 & 120 & 0.0001 & 0.0001 \\
\cline{1-4}
12 & 130 & 0.0001 & 0.0000 \\
\cline{1-4}
13 & 140 & 0.0001 & 0.0000 \\
\cline{1-4}
14 & 150 & 0.0001 & 0.0001 \\
\cline{1-4}
15 & 160 & 0.0001 & 0.0001 \\
\cline{1-4}
16 & 170 & 0.0002 & 0.0002 \\
\cline{1-4}
17 & 180 & 0.0002 & 0.0002 \\
\cline{1-4}
18 & 190 & 0.0002 & 0.0002 \\
\cline{1-4}
19 & 200 & 0.0002 & 0.0001 \\
\cline{1-4}
20 & 210 & 0.0002 & 0.0000 \\
\cline{1-4}
21 & 220 & 0.0002 & 0.0002 \\
\cline{1-4}
22 & 230 & 0.0002 & 0.0000 \\
\cline{1-4}
23 & 240 & 0.0002 & 0.0002 \\
\cline{1-4}
24 & 250 & 0.0002 & 0.0002 \\
\cline{1-4}
25 & 260 & 0.0002 & 0.0002 \\
\cline{1-4}
26 & 270 & 0.0003 & 0.0003 \\
\cline{1-4}
27 & 280 & 0.0003 & 0.0003 \\
\cline{1-4}
28 & 290 & 0.0003 & 0.0003 \\
\cline{1-4}
29 & 300 & 0.0003 & 0.0003 \\
\cline{1-4}
30 & 310 & 0.0003 & 0.0001 \\
\cline{1-4}
31 & 320 & 0.0003 & 0.0003 \\
\cline{1-4}
32 & 330 & 0.0003 & 0.0003 \\
\cline{1-4}
33 & 340 & 0.0003 & 0.0003 \\
\cline{1-4}
34 & 350 & 0.0003 & 0.0003 \\
\cline{1-4}
35 & 360 & 0.0004 & 0.0004 \\
\cline{1-4}
36 & 370 & 0.0004 & 0.0004 \\
\cline{1-4}
37 & 380 & 0.0004 & 0.0001 \\
\cline{1-4}
38 & 390 & 0.0004 & 0.0001 \\
\cline{1-4}
39 & 400 & 0.0004 & 0.0004 \\
\cline{1-4}
\end{tabular}
\end{adjustbox}
\end{table}

\begin{table}
\centering
\begin{adjustbox}{width=1\textwidth/2}
\begin{tabular}{|c|c|c|c|}
\hline
 & Dimensione (n) & Tempo (s) & Deviazione standard \\
40 & 410 & 0.0004 & 0.0004 \\
\cline{1-4}
41 & 420 & 0.0004 & 0.0001 \\
\cline{1-4}
42 & 430 & 0.0004 & 0.0004 \\
\cline{1-4}
43 & 440 & 0.0004 & 0.0002 \\
\cline{1-4}
44 & 450 & 0.0005 & 0.0005 \\
\cline{1-4}
45 & 460 & 0.0005 & 0.0001 \\
\cline{1-4}
46 & 470 & 0.0005 & 0.0001 \\
\cline{1-4}
47 & 480 & 0.0005 & 0.0005 \\
\cline{1-4}
48 & 490 & 0.0005 & 0.0005 \\
\cline{1-4}
49 & 500 & 0.0005 & 0.0005 \\
\cline{1-4}
50 & 510 & 0.0005 & 0.0005 \\
\cline{1-4}
51 & 520 & 0.0006 & 0.0006 \\
\cline{1-4}
52 & 530 & 0.0006 & 0.0006 \\
\cline{1-4}
53 & 540 & 0.0006 & 0.0003 \\
\cline{1-4}
54 & 550 & 0.0006 & 0.0006 \\
\cline{1-4}
55 & 560 & 0.0006 & 0.0006 \\
\cline{1-4}
56 & 570 & 0.0006 & 0.0006 \\
\cline{1-4}
57 & 580 & 0.0006 & 0.0006 \\
\cline{1-4}
58 & 590 & 0.0006 & 0.0001 \\
\cline{1-4}
59 & 600 & 0.0007 & 0.0003 \\
\cline{1-4}
60 & 610 & 0.0007 & 0.0002 \\
\cline{1-4}
61 & 620 & 0.0007 & 0.0007 \\
\cline{1-4}
62 & 630 & 0.0007 & 0.0007 \\
\cline{1-4}
63 & 640 & 0.0007 & 0.0007 \\
\cline{1-4}
64 & 650 & 0.0008 & 0.0008 \\
\cline{1-4}
65 & 660 & 0.0008 & 0.0008 \\
\cline{1-4}
66 & 670 & 0.0007 & 0.0007 \\
\cline{1-4}
67 & 680 & 0.0008 & 0.0008 \\
\cline{1-4}
68 & 690 & 0.0008 & 0.0008 \\
\cline{1-4}
69 & 700 & 0.0008 & 0.0008 \\
\cline{1-4}
70 & 710 & 0.0008 & 0.0008 \\
\cline{1-4}
71 & 720 & 0.0008 & 0.0008 \\
\cline{1-4}
72 & 730 & 0.0009 & 0.0009 \\
\cline{1-4}
73 & 740 & 0.0009 & 0.0009 \\
\cline{1-4}
74 & 750 & 0.0009 & 0.0009 \\
\cline{1-4}
75 & 760 & 0.0009 & 0.0009 \\
\cline{1-4}
76 & 770 & 0.0009 & 0.0009 \\
\cline{1-4}
77 & 780 & 0.0009 & 0.0009 \\
\cline{1-4}
78 & 790 & 0.0012 & 0.0008 \\
\cline{1-4}
79 & 800 & 0.0010 & 0.0005 \\
\cline{1-4}
\end{tabular}
\end{adjustbox}
\end{table}

\begin{table}
\centering
\begin{adjustbox}{width=1\textwidth/2}
\begin{tabular}{|c|c|c|c|}
\hline
 & Dimensione (n) & Tempo (s) & Deviazione standard \\
80 & 810 & 0.0011 & 0.0003 \\
\cline{1-4}
81 & 820 & 0.0010 & 0.0007 \\
\cline{1-4}
82 & 830 & 0.0011 & 0.0011 \\
\cline{1-4}
83 & 840 & 0.0011 & 0.0011 \\
\cline{1-4}
84 & 850 & 0.0011 & 0.0009 \\
\cline{1-4}
85 & 860 & 0.0011 & 0.0008 \\
\cline{1-4}
86 & 870 & 0.0011 & 0.0009 \\
\cline{1-4}
87 & 880 & 0.0011 & 0.0011 \\
\cline{1-4}
88 & 890 & 0.0013 & 0.0013 \\
\cline{1-4}
89 & 900 & 0.0015 & 0.0015 \\
\cline{1-4}
90 & 910 & 0.0011 & 0.0011 \\
\cline{1-4}
91 & 920 & 0.0011 & 0.0011 \\
\cline{1-4}
92 & 930 & 0.0012 & 0.0012 \\
\cline{1-4}
93 & 940 & 0.0013 & 0.0013 \\
\cline{1-4}
94 & 950 & 0.0014 & 0.0012 \\
\cline{1-4}
95 & 960 & 0.0012 & 0.0006 \\
\cline{1-4}
96 & 970 & 0.0014 & 0.0006 \\
\cline{1-4}
97 & 980 & 0.0012 & 0.0008 \\
\cline{1-4}
98 & 990 & 0.0013 & 0.0013 \\
\cline{1-4}
\end{tabular}
\end{adjustbox}
\end{table}


\end{document}
